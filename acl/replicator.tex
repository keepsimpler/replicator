% This must be in the first 5 lines to tell arXiv to use pdfLaTeX, which is strongly recommended.
\pdfoutput=1
% In particular, the hyperref package requires pdfLaTeX in order to break URLs across lines.

\documentclass[11pt]{article}

% Remove the "review" option to generate the final version.
\usepackage[review]{acl}

% Standard package includes
\usepackage{times}
\usepackage{latexsym}

% For proper rendering and hyphenation of words containing Latin characters (including in bib files)
\usepackage[T1]{fontenc}
% For Vietnamese characters
% \usepackage[T5]{fontenc}
% See https://www.latex-project.org/help/documentation/encguide.pdf for other character sets

% This assumes your files are encoded as UTF8
\usepackage[utf8]{inputenc}

% This is not strictly necessary, and may be commented out,
% but it will improve the layout of the manuscript,
% and will typically save some space.
\usepackage{microtype}

% If the title and author information does not fit in the area allocated, uncomment the following
%
%\setlength\titlebox{<dim>}
%
% and set <dim> to something 5cm or larger.

\title{Instructions for *ACL Proceedings}

% Author information can be set in various styles:
% For several authors from the same institution:
% \author{Author 1 \and ... \and Author n \\
%         Address line \\ ... \\ Address line}
% if the names do not fit well on one line use
%         Author 1 \\ {\bf Author 2} \\ ... \\ {\bf Author n} \\
% For authors from different institutions:
% \author{Author 1 \\ Address line \\  ... \\ Address line
%         \And  ... \And
%         Author n \\ Address line \\ ... \\ Address line}
% To start a seperate ``row'' of authors use \AND, as in
% \author{Author 1 \\ Address line \\  ... \\ Address line
%         \AND
%         Author 2 \\ Address line \\ ... \\ Address line \And
%         Author 3 \\ Address line \\ ... \\ Address line}

\author{First Author \\
  Affiliation / Address line 1 \\
  Affiliation / Address line 2 \\
  Affiliation / Address line 3 \\
  \texttt{email@domain} \\\And
  Second Author \\
  Affiliation / Address line 1 \\
  Affiliation / Address line 2 \\
  Affiliation / Address line 3 \\
  \texttt{email@domain} \\}

\begin{document}
\maketitle
\begin{abstract}
  
  The replicator equations illustrate the evolution of the distribution of types in a population. We utilize replicator dynamics as the backbone to create an architecture, Replicator, which directly learns discrete probability distributions over feature spaces. Replicator could apply whenever needing to learn probability distributions. In Replicator, data features first transform into discrete distributions over a probability space. After that, the discrete distributions evolve according to the replicator equation rules. Finally, the discrete distributions are converted back to the original data space. Replicator achieved the-state-of-art accuracy and performance in both synthetic and realistic datasets.
\end{abstract}

\section{Introduction}

\section{Approach}

% Language modeling is usually framed as unsupervised distribution estimation for a set of 

Given a sequence of words $(x_1, x_2, \ldots, x_T)$, the goal of a language model is to estimate the joint probability of the sequence:
\begin{equation}
  P(x_1, x_2, \ldots, x_T).
\end{equation}
Since language has a sequential ordering, it is common to factorize the joint probability of the sequence as the product of conditional probabilities:
\begin{equation}
  P(x_1, x_2, \ldots, x_T) = \prod_{t=1}^n P(x_t \vert x_1, \ldots, x_{t-1}).
\end{equation}

Thus, the efficient modeling of conditional probabilities of the next
word given all the previous ones, $P(x_t \vert x_1, \ldots, x_{t-1})$,
is critical for the capability of the language model.

Although the neural models computing these conditional probabilities have improved their expressiveness significantly, such as self-attention architectures like the Transformer \cite{vaswani2017attention}, they all have one limitation.

We here first using replicator dynamics to directly model the conditional probabilities.

\subsection{Replicator Equation}
In the field of evolutionary game theory, the replicator equation is defined as follows. Let $x_i$ is the proportion of type $i$ in the population, $(x_1, x_2, \ldots, x_n)$ is the vector of the distribution of types in the population, and $\sum_{i=1}^n x_i = 1$. The continuous form of replicator equation can be written in the form:
\begin{equation}
  \dot{x_i} = x_i \left((Ax)_i - x^T A x \right) ,
\end{equation}
where the payoff matrix $A$ holds all the fitness information for the population: the fitness of type $i$ is $(Ax)_i$ and the average fitness of the population is $x^T A x$.

Similar to the continuous form, the discrete form of the replicator equation can be written as:
\begin{equation}
  x_i^{'} = x_i + x_i \left[ (Ax)_i - x^T A x \right] .
\end{equation}

\subsection{Bayesian Inference}


\subsection{Bayesian Inference is a special case of Replicator Equation}


\cite{harper2009replicator}

To the best of our knowledge, however, the Replicator is the first language model relying entirely on the replicator equation to compute the conditional probabilities over the words.

\section{Model Architecture}

Our model does not require position embeddings because the token-mixing Replicator Layer are sensitive to the order of the input tokens. %such information will be captured in $P$

Spatial information acrosßs tokens (words), which position the words should be. 

Our model accepts a word sequence, linearly embeds each word to a feature vector, and obtain a 2-rank tensor with shape (words $\times$ features) as input. Replicator network (architecture) makes use of two types of Replicator layers: word Replicators and feature Replicators. The word Replicators capture co-occurrences patterns among words in the sequence. They operate on each feature independently and take individual columns as inputs. The feature Replicators allow communication between different features.

\section{Why Replicator}
Dynamic directly in probability space, has no vanishing / exploding gradient problems, thus no normal layer is required.


\section{Document Body}

\subsection{Footnotes}

Footnotes are inserted with the \verb|\footnote| command.\footnote{This is a footnote.}

\subsection{Tables and figures}

See Table~\ref{tab:accents} for an example of a table and its caption.
\textbf{Do not override the default caption sizes.}

\begin{table}
\centering
\begin{tabular}{lc}
\hline
\textbf{Command} & \textbf{Output}\\
\hline
\verb|{\"a}| & {\"a} \\
\verb|{\^e}| & {\^e} \\
\verb|{\`i}| & {\`i} \\ 
\verb|{\.I}| & {\.I} \\ 
\verb|{\o}| & {\o} \\
\verb|{\'u}| & {\'u}  \\ 
\verb|{\aa}| & {\aa}  \\\hline
\end{tabular}
\begin{tabular}{lc}
\hline
\textbf{Command} & \textbf{Output}\\
\hline
\verb|{\c c}| & {\c c} \\ 
\verb|{\u g}| & {\u g} \\ 
\verb|{\l}| & {\l} \\ 
\verb|{\~n}| & {\~n} \\ 
\verb|{\H o}| & {\H o} \\ 
\verb|{\v r}| & {\v r} \\ 
\verb|{\ss}| & {\ss} \\
\hline
\end{tabular}
\caption{Example commands for accented characters, to be used in, \emph{e.g.}, Bib\TeX{} entries.}
\label{tab:accents}
\end{table}

\subsection{Hyperlinks}

Users of older versions of \LaTeX{} may encounter the following error during compilation: 
\begin{quote}
\tt\verb|\pdfendlink| ended up in different nesting level than \verb|\pdfstartlink|.
\end{quote}
This happens when pdf\LaTeX{} is used and a citation splits across a page boundary. The best way to fix this is to upgrade \LaTeX{} to 2018-12-01 or later.

\subsection{Citations}

\begin{table*}
\centering
\begin{tabular}{lll}
\hline
\textbf{Output} & \textbf{natbib command} & \textbf{Old ACL-style command}\\
\hline
\citep{Gusfield:97} & \verb|\citep| & \verb|\cite| \\
\citealp{Gusfield:97} & \verb|\citealp| & no equivalent \\
\citet{Gusfield:97} & \verb|\citet| & \verb|\newcite| \\
\citeyearpar{Gusfield:97} & \verb|\citeyearpar| & \verb|\shortcite| \\
\hline
\end{tabular}
\caption{\label{citation-guide}
Citation commands supported by the style file.
The style is based on the natbib package and supports all natbib citation commands.
It also supports commands defined in previous ACL style files for compatibility.
}
\end{table*}

Table~\ref{citation-guide} shows the syntax supported by the style files.
We encourage you to use the natbib styles.
You can use the command \verb|\citet| (cite in text) to get ``author (year)'' citations, like this citation to a paper by \citet{Gusfield:97}.
You can use the command \verb|\citep| (cite in parentheses) to get ``(author, year)'' citations \citep{Gusfield:97}.
You can use the command \verb|\citealp| (alternative cite without parentheses) to get ``author, year'' citations, which is useful for using citations within parentheses (e.g. \citealp{Gusfield:97}).

\subsection{References}

\nocite{vaswani2017attention}

The \LaTeX{} and Bib\TeX{} style files provided roughly follow the American Psychological Association format.
If your own bib file is named \texttt{custom.bib}, then placing the following before any appendices in your \LaTeX{} file will generate the references section for you:
\begin{quote}
\begin{verbatim}
\bibliographystyle{acl_natbib}
\bibliography{custom}
\end{verbatim}
\end{quote}

You can obtain the complete ACL Anthology as a Bib\TeX{} file from \url{https://aclweb.org/anthology/anthology.bib.gz}.
To include both the Anthology and your own .bib file, use the following instead of the above.
\begin{quote}
\begin{verbatim}
\bibliographystyle{acl_natbib}
\bibliography{anthology,custom}
\end{verbatim}
\end{quote}

Please see Section~\ref{sec:bibtex} for information on preparing Bib\TeX{} files.

\subsection{Appendices}

Use \verb|\appendix| before any appendix section to switch the section numbering over to letters. See Appendix~\ref{sec:appendix} for an example.

\section{Bib\TeX{} Files}
\label{sec:bibtex}

Unicode cannot be used in Bib\TeX{} entries, and some ways of typing special characters can disrupt Bib\TeX's alphabetization. The recommended way of typing special characters is shown in Table~\ref{tab:accents}.

Please ensure that Bib\TeX{} records contain DOIs or URLs when possible, and for all the ACL materials that you reference.
Use the \verb|doi| field for DOIs and the \verb|url| field for URLs.
If a Bib\TeX{} entry has a URL or DOI field, the paper title in the references section will appear as a hyperlink to the paper, using the hyperref \LaTeX{} package.

\section*{Acknowledgements}

This document has been adapted
by Steven Bethard, Ryan Cotterell and Rui Yan
from the instructions for earlier ACL and NAACL proceedings, including those for 
ACL 2019 by Douwe Kiela and Ivan Vuli\'{c},
NAACL 2019 by Stephanie Lukin and Alla Roskovskaya, 
ACL 2018 by Shay Cohen, Kevin Gimpel, and Wei Lu, 
NAACL 2018 by Margaret Mitchell and Stephanie Lukin,
Bib\TeX{} suggestions for (NA)ACL 2017/2018 from Jason Eisner,
ACL 2017 by Dan Gildea and Min-Yen Kan, 
NAACL 2017 by Margaret Mitchell, 
ACL 2012 by Maggie Li and Michael White, 
ACL 2010 by Jing-Shin Chang and Philipp Koehn, 
ACL 2008 by Johanna D. Moore, Simone Teufel, James Allan, and Sadaoki Furui, 
ACL 2005 by Hwee Tou Ng and Kemal Oflazer, 
ACL 2002 by Eugene Charniak and Dekang Lin, 
and earlier ACL and EACL formats written by several people, including
John Chen, Henry S. Thompson and Donald Walker.
Additional elements were taken from the formatting instructions of the \emph{International Joint Conference on Artificial Intelligence} and the \emph{Conference on Computer Vision and Pattern Recognition}.

% Entries for the entire Anthology, followed by custom entries
\bibliography{anthology,custom}
\bibliographystyle{acl_natbib}

\appendix

\section{Example Appendix}
\label{sec:appendix}

This is an appendix.

\end{document}
